% !TeX encoding = UTF-8
% !TeX spellcheck = fr_CA

% nohyperref pour charger le package hyperref manuellement 
% voir https://gitlab.com/vigou3/ulthese/issues/2
\documentclass[PhD,nohyperref,english,french]{ulthese}
%% ===================== packages ============================================
\ifxetex\else \usepackage[utf8]{inputenc} \fi
\usepackage{amsmath,amsfonts,amssymb,mathtools,siunitx,etoolbox}
\usepackage{hyperref} % dernier package chargé sauf glossaries
\usepackage[noredefwarn]{glossaries} % aucun warning pour remplacements 
% de la classe memoir
\usepackage{glosnomtools}
%% ===================== preamble ============================================
%% ------------- glossaries ------------------------------------------------
\makeglossaries
\setglossarystyle{nomencl-fr-en}
\setacronymstyle{fr-acr}
\loadglsentries{sample_glosnomtools_glos}
\glssetwidest{LOOP}  % IMPORTANT : entrée la plus large de la nomenclature
%% ------------- ulthese ----------------------------------------------------
\titre{Exemple d'une nomenclature mathématique avec \texttt{glossaries} et \texttt{glossnomtools} dans \texttt{ulthese}}
\auteur{Francis Gagnon}
\direction{André Desbiens, directeur de recherche}
\programme{Doctorat en génie électrique}
\annee{2018}
%% ============================================================================

\begin{document}
	
\frontmatter                    % pages liminaires

\frontispice

\begin{SingleSpace}
	\abnormalparskip{0pt}
	\printglossary[title=Nomenclature]
\end{SingleSpace}

\mainmatter                     % corps du document

\chapter{Exemple d'utilisation de \texttt{glosnomtools}}

\section{Compilation}
Le package \texttt{glossaries} est inclus dans MikTeX et TeX Live. Un peu comme une bibliographie, la compilation d'un glossaire ce fait en exécutant les commandes dans l'ordre suivant :
\begin{enumerate}
	\item \texttt{pdflatex}
	\item \texttt{makeglossaries}
	\item \texttt{pdflatex}
\end{enumerate}
La commande \texttt{makeglossaries} est directement accessible dans les menu de TeXstudio ou avec \texttt{F9}. Pour un autre éditeur, il faut ajouter une commande personnalisée. Par exemple, dans TexMaker, c’est accessible dans le menu \texttt{Utilisateur > Commandes Utilisateur > Éditer Commandes Utilisateur}:
\begin{itemize}
	\item \texttt{Item menu : glossaries} 
	\item \texttt{Commande : makeglossaries \%}
\end{itemize}
Sous Windows avec MiKTeX, il est possible que le script \texttt{makeglossaries} nécessite une installation de Perl (voir \url{https://tinyurl.com/ybnoyqjp}). Une fois Perl installé, il faut exécuter le script \texttt{perltex.exe} disponible dans le répertoire d'installation de MiKTeX.

\section{Symboles, indices et accentuations}
L'insertion d'un symbole mathématique simple avec hyperlien se fait avec la macro \texttt{gls} de glossaries : \gls{k}, \gls{mat.A} et \gls{mat.b}. C'est la même chose pour les abréviations : \gls{LOOP}. Les abréviations sont uniquement définies à leur première utilisation : le \gls{LOOP}.

Les indices doivent être définis avec un label du format \texttt{sub.resteDuLabel}: l'indice \gls{sub.a} désigne l'air. La macro \texttt{glsub} permet ensuite d'ajouter un indice à une variable : \glsub{d}{v}, \glsub{z}{v}, \glsub{T}{v}, \glsub{D}{a}, \glsub{rho}{w} et \glsub{mu}{v}. En définissant les symboles mathématiques avec \texttt{newglosentrymath}, toutes les macros peuvent s'utiliser autant en mode \texttt{text} qu'en mode \texttt{math} (avec \texttt{\$\$}). Par contre, je crois personnellement qu'il est mieux d'écrire explicitement les symboles dans une équation (pas d'hyperlien):
\begin{equation}
d"v + \glsub{d}{v} = \SI{10.0}{\centi\meter} = 3.937\qtmark
\end{equation}
En mode \texttt{math}, j'ai configuré le caractère \texttt{<">} comme raccourci pour l'écriture d'un indice sans italique (\texttt{<\_>} pour indice en italique) \footnote{ La macro \texttt{qtmark} permet d'insérer le caractère \texttt{<">}. En ISO, les indices représentant l'abréviation d'un mot s'écrivent sans italique, et, celles représentant une variable, en italique.}. La macro \texttt{glsvi} permet d'ajouter une variable en indice à une autre variable : \glsvi{T}{k}. Il est aussi possible d'ajouter deux indices séparés par une virgule avec \texttt{glsubs} : \glsubs{D}{w}{a}. En l’occurrence, l'opérateur virgule est ajouté dans la nomenclature, qui doit donc être défini dans le glossaire sous le label \texttt{op.comma}.

Il est aussi possible d'ajouter des accents sur les variables : \glsdot{m} et \glsbar{T}. À leur utilisation respective, un opérateur est ajouté dans la nomenclature. Ils doivent donc être définis à leur label respectif. Les macros disponibles pour accentuer une variable sont:
\begin{itemize} 
	\item \texttt{glsdot} (définition de l'accent au label \texttt{op.dot})
	\item \texttt{glsbar} (définition de l'accent au label \texttt{op.bar})
	\item \texttt{glshat} (définition de l'accent au label \texttt{op.hat})
	\item \texttt{glstilde} (définition de l'accent au label \texttt{op.tilde})
\end{itemize}
De plus, un argument optionnel permet d'ajouter des accents à toutes les macros précédentes : \glsub[bar]{T}{v} et \glsubs[dot]{m}{v}{a}. Il y a deux arguments optionnels dans le cas de \texttt{glsvi} : \glsvi[dot]{m}{k}, \glsvi[][dot]{k}{m} et \glsvi[bar][dot]{T}{m}.

\section{Langue, abréviations et nomenclature}

S'il y a des changements de langues à travers le document, il faut changer de style pour les acronymes. La clé \texttt{decriptionfr} doit être préalablement définie dans le glossaire. Par la suite, il y a trois options de style disponibles : 
\begin{itemize}
	\setacronymstyle{fr-en-acr} \item bilingue  avec \texttt{fr-en-acr}  : \acrfull{ODE}
	\setacronymstyle{en-acr} \item anglais avec \texttt{en-acr} : \acrfull{ODE}
	\setacronymstyle{fr-acr}  \item français avec \texttt{fr-acr} : \acrfull{ODE}
\end{itemize}
La macro \texttt{glsfren} affiche l'acronyme en bilingue, peu importe le style configuré : \glsfren{ODE}. Les descriptions en français sont aussi accessibles avec \texttt{glsdescfr} : les \glsdescfr{ODE}. De manière similaire, il est possible de changer la langue de la nomenclature avec trois options de style:
\begin{itemize}
	\item descriptions bilingues avec \texttt{nomencl-fr-en}
	\item descriptions en anglais avec \texttt{nomencl-en}
	\item descriptions en français avec \texttt{nomencl-fr}
\end{itemize}
L'exemple de la page ii est une nomenclature bilingue. La clé \texttt{symbol} ne contient pas le symbole mathématique, mais bien les unités affichées dans la nomenclature.

\end{document}